\documentclass[a4paper,11pt,english,plainsub]{uioexam}
\usepackage[utf8]{inputenc}
\usepackage[T1]{fontenc}
\usepackage{babel,textcomp}
\usepackage{amsmath,amsfonts} 
\usepackage{graphics}
\usepackage{listings}

\newcommand{\ttc}[1]{\choice {\tt #1}}

\emne{STK-INF3000/4000}{Selected topics in data science}
\dato{June 13, 2016}
\tid{14.30}{18.30}
\hjelpemidler{\begin{tabular}[t]{@{}l}Calculator\\
\end{tabular}}

\begin{document}

\tableofcontents

\bigskip

%\begin{flushright}
%\sc Kandidatnr. \ \underline{\hspace*{2cm}}
%\end{flushright}

%\begin{flushright}
%\sc Candidate no:\ \underline{\hspace*{2cm}}
%\end{flushright}


\oppgave{Python}

\deloppgave{Lists}

What will be the output of the python code listed below?

\begin{lstlisting}[language=Python]
print range(10)[1:5]  
\end{lstlisting}

\begin{choicelist}
  \choice {\tt [1, 2, 3, 4, 5]}
  \choice {\tt [0, 1, 2, 3, 4, 5]}
  \choice {\tt [1, 2, 3, 4]}
  \choice {\tt [0, 1, 2, 3, 4]}
\end{choicelist}

\deloppgave{Dictionaries}

What will be the output of the python code listed below?

\begin{lstlisting}[language=Python]
print {i: i**2 for i in range(10) if i % 2}
\end{lstlisting}

\begin{choicelist}
  \choice {\tt [1, 9, 25, 49, 81]}
  \choice {\tt {1: 1, 3: 9, 5: 25, 7: 49, 9: 81}}
  \choice {\tt {0: 0, 8: 64, 2: 4, 4: 16, 6: 36}}
  \choice {\tt {1: 1, 3: 9, 5: 25, 7: 49}}
\end{choicelist}

\oppgave{Data Processing}

\deloppgave{Data Frame Transformation}

Given a {\tt pandas} data frame {\tt a}, given by

\begin{verbatim}
  category label  value
0     high     a      0
1      low     b      1
2     high     c      2
3      low     a      3
4     high     b      4
5      low     c      5
\end{verbatim}

what command can be used to transform it into the following form?

\begin{verbatim}
category  high  low
label
a            0    3
b            4    1
c            2    5
\end{verbatim}

\begin{choicelist}
  \choice {\tt a.stack}
  \choice {\tt a.groupby}
  \choice {\tt a.pivot}
\end{choicelist}

\deloppgave{Data Frame Transformation}

Given a {\tt pandas} data frame {\tt a}, given by

\begin{verbatim}
  category label  value
0     high     a      0
1      low     b      1
2     high     c      2
3      low     a      3
4     high     b      4
5      low     c      5
\end{verbatim}

what command can be used to transform it into the following form?

\begin{verbatim}
          value
category
high          6
low           9
\end{verbatim}

\begin{choicelist}
  \choice {\tt a.stack}
  \choice {\tt a.groupby}
  \choice {\tt a.pivot}
\end{choicelist}

\oppgave{Apache Spark}

What output will the following Apache Spark code produce (a Spark
Context is assumed to exist as {\tt sc}).

\begin{lstlisting}[language=Python]
  print (sc
    .parallelize(range(10))
    .map(lambda x: [x % 2, x])
    .reduceByKey(lambda x, y: x + y)
    .collect())
\end{lstlisting}

\begin{choicelist}
  \ttc{[(0, 20), (1, 20)]}
  \ttc{[(0, 20), (1, 25)]}
  \ttc{[(0, 10), (1, 15), (3, 10)]}
\end{choicelist}

\oppgave{Machine learning}

\deloppgave{Decision trees}

Which of the following statements about decision trees hold true?

\begin{choicelist}
  \choice Trees of very low depth tend to show high bias.
  \choice Variable importance can be estimated from the  splitting
  points chosen by the algorithm and resulting information gain.
  \choice Trees of excessive depth tend to show high variance.
\end{choicelist}


\end{document}